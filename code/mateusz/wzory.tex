%%%%%%%%%%%%%%%%%%%%%%%%%%%%%%%%%%%%%%%%%%%%%%%%%%%%%%%%%%%%%%%%%%%%%%%%%%%%%%%%

\usepackage[utf8]{inputenc}
\usepackage{polski}
\usepackage{amsfonts}
\usepackage{amsmath}
\DeclareMathOperator{\arcsinh}{arcsinh}
\DeclareMathOperator{\dx}{ dx}

%%%%%%%%%%%%%%%%%%%%%%%%%%%%%%%%%%%%%%%%%%%%%%%%%%%%%%%%%%%%%%%%%%%%%%%%%%%%%%%%

Długość wykresu funkcji $f : [a, b] \to \mathbb{R}$:
  \hfill
  $\int_a^b \sqrt{f'(x)^2 + 1} \; dx$.

Pole figury obrotowej $f : [a, b] \to \mathbb{R}$:
  \hfill
  $2 \pi \int_a^b |f(x)| \sqrt{f'(x)^2 + 1} \; dx$.

Objętość figury obrotowej $f : [a, b] \to \mathbb{R}$:
  \hfill
  $\pi \int_a^b f(x)^2 \; dx$.

$\int \sqrt{x^2 + 1} \; dx = \frac12 \left( x\sqrt{x^2 + 1} + \arcsinh x  \right) + c$

$\int \sqrt{x^2 + 1} \dx\ =\ \frac12 \left( x\sqrt{x^2 + 1} + \arcsinh x \right) + c$

 \( \int \sqrt{x^2 + 1} \dx\ =\ \frac12 \left( x\sqrt{x^2 + 1} + \arcsinh x \right) + c\) 
  \ \ \ $(\arcsinh = \mathtt{asinh})$ \\
  \( \int \sqrt{1 - x^2} \dx\ =\ \frac12 \left( x\sqrt{1 - x^2} + \arcsin x \right) + c\) \\
  \( \int \frac{1}{ax^2 + bx + c} \dx\ =\ \frac{2}{\sqrt{4ac - b^2}} \arctan
      \frac{2ax + b}{\sqrt{4ac - b^2}} \qquad \qquad (\Delta < 0)\) \\
  \( \int \frac{x}{ax^2 + bx + c} \dx\ =\ \frac{1}{2a} \ln |ax^2 + bx + c|
      - \frac{b}{2a} \int \frac{\dx}{ax^2 + bx + c} \) \\
  \( \int \tan x \dx = - \ln |\cos x| + c \) \\
  \( (\arcsin x)' = \frac{1}{\sqrt{1 - x^2}}, \qquad \qquad
     (\arccos x)' = -\frac{1}{\sqrt{1 - x^2}} \) \\
  \( \frac1{\pi} = 0.31831,\ \ \pi^2 = 9.86960,\ \ \frac1{\pi^2} = 0.10132,\ \ \frac1e = 0.36788,
  \ \ \gamma = 0.577215664901532\) \\
  \( H_n = \ln n + \gamma + \frac1{2n} - \frac1{12n^2} + O(n^{-4}) \) \\
  \( \ln n! = n \ln n - n + \frac12 \ln(2 \pi n) + \frac1{12n} - \frac1{360n^3} + \frac1{1260n^5}
  - O(n^{-7}) \)

$(\tan x)'\ =\ \frac{1}{\cos^2 x}\ =\ 1 + \tan^2 x$

$(\arcsin x)'\ =\ \frac{1}{\sqrt{1-x^2}} \hfill
	(\arccos x)'\ =\ \frac{-1}{\sqrt{1-x^2}}$

$(\arctan x)'\ =\ \frac{1}{1+x^2} \hfill
	(\arctan2 x)'\ =\ \frac{-1}{1+x^2}$

Wzór Picka:
  \hfill
  $POLE = WEW + \frac{1}{2} \cdot BRZEG  - 1$

Tw. sinusów:
  \hfill
  $\frac{a}{\sin \alpha} = \frac{b}{\sin \beta} = \frac{c}{\sin \omega} = 2 \cdot R$

Tw. cosinusów:
  \hfill
  $a^2 = b^2 + c^2 - 2 \cdot b \cdot c \cdot \cos \alpha$

$R = \frac{a \cdot b \cdot c}{4 \cdot POLE}$

Jeśli $n=2^{a_0} \cdot {p_1}^{a_1} \cdot ... \cdot {p_r}^{a_r} \cdot {q_1}^{b_1} \cdot ... \cdot {q_s}^{b_s}$ i $B= \prod (b_i+1)$, gdzie $p_i$ to liczby postaci $4 \cdot k + 3$, a $q_i$ to liczby postaci $4 \cdot k + 1$, to liczba sposobów na zapisanie $n$ jako sumy dwóch kwadratów liczb naturalnych wynosi:
      \begin{itemize}
        \item 0, jeśli któraś z liczb $a_i$ nie jest całkowita
        \item $\frac{B}{2}$, jeśli $B$ jest parzyste
        \item $\frac{B-(-1)^{a_0}}{2}$, jeśli $B$ jest nieparzyste
      \end{itemize}

%%%%%%%%%%%%%%%%%%%%%%%%%%%%%%%%%%%%%%%%%%%%%%%%%%%%%%%%%%%%%%%%%%%%%%%%%%%%%%%%
