%%%%%%%%%%%%%%%%%%%%%%%%%%%%%%%%%%%%%%%%%%%%%%%%%%%%%%%%%%%%%%%%%%%%%%%%%%%%%%%%

\usepackage[utf8]{inputenc}
\usepackage{polski}
\usepackage{amsfonts}

%%%%%%%%%%%%%%%%%%%%%%%%%%%%%%%%%%%%%%%%%%%%%%%%%%%%%%%%%%%%%%%%%%%%%%%%%%%%%%%%

Niech $M(n) = \sum_{i = 1}^n \mu(i)$.
Można policzyć $M(n)$ w $O \left( n^{2/3} \cdot log(\textrm{smth}) \right)$.
Dla $u = n^{1 / 3}$, wystarczy spreprocesować $M$ do $n^{2 / 3}$
  i obliczyć $M(n)$ wzorem:

\[
  M(n) = M(u) - \sum_{m = 1}^u \mu(m)
      \sum_{i = \left\lfloor \frac{u}{m}
            \right\rfloor + 1}^{\left\lfloor \frac{n}{m} \right\rfloor}
         M \left( \left\lfloor \frac{n}{mi} \right\rfloor \right).
\]

%%%%%%%%%%%%%%%%%%%%%%%%%%%%%%%%%%%%%%%%%%%%%%%%%%%%%%%%%%%%%%%%%%%%%%%%%%%%%%%%
